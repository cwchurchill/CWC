%% manuscript produces a one-column, double-spaced document:
%\documentclass[manuscript]{aastex}

\documentclass{emulateapj}
\usepackage{apjfonts}
\usepackage{iondefs}
%\documentclass[12pt,preprint]{aastex}
%\usepackage{color}
%This is how you implement \textcolor{red}{colour} in a LaTeX document.
%% preprint style produces a one-column, single-spaced document.

%% preprint2 produces a double-column, single-spaced document:
%\documentclass[preprint2]{aastex}

% personal LaTex mark--ups
%
%        **** DO NOT MODIFY BELOW ****   **** DO NOT MODIFY BELOW **** 
%

%% You can insert a short comment on the title page using the command below.
%\slugcomment{ApJ Draft }
\shorttitle{\sc {\MgII}-Absorber Galaxy Masses}
\shortauthors{\sc Churchill et~al.}

%% This is the end of the preamble.  Indicate the beginning of the
%% paper itself with \begin{document}.

\begin{document}

%% LaTeX will automatically break titles if they run longer than
%% one line. However, you may use \\ to force a line break if
%% you desire.

\title{The Dependence of {\MgII} Absorption on Galaxy Mass}

%% Use \author, \affil, and the \and command to format
%% author and affiliation information.
%% Note that \email has replaced the old \authoremail command
%% from AASTeX v4.0. You can use \email to mark an email address
%% anywhere in the paper, not just in the front matter.
%% As in the title, use \\ to force line breaks.

\author{\sc
Christopher W. Churchill\altaffilmark{2}
Sebastian Trujillo-Gomez\altaffilmark{1}
Nikole M. Nielsen\altaffilmark{1}
and
Glenn G. Kacprzak\altaffilmark{2,3}
}
                                                                                
\altaffiltext{1}{New Mexico State University, Las Cruces, NM 88003}
\altaffiltext{2}{Swinburne University of Technology, Victoria 3122, Australia}
\altaffiltext{3}{Australian Research Council Super Science Fellow}

\begin{abstract}

We present an azimuthal angle bi-modality in the distribution of gas
around galaxies as traced by {\MgII} absorption: Halo gas prefers to

\end{abstract}



\keywords{galaxies: halos --- galaxies: intergalactic medium ---
  quasars: absorption lines}

\section{Introduction}
\label{sec:intro}

It is well established that {\MgII} absorption, detected in background
quasar/galaxy spectra, arises from gas associated with foreground
galaxies and provides a unique means to directly observe mechanisms by
which galaxies acquire, chemically enrich, recycle, and expel their
gaseous component \citep[see][for a review]{cwc-china}.  The
{\MgIIdblt} absorption doublet is an ideal probe since it traces
low-ionization gas with $10^{16} \leq N(\HI) \leq 10^{22}$~{\cmsq}
\citep{archiveI,weakII} and, as a result, it is detected out to
projected galactic radii of $\sim 100$ kpc
\citep{kacprzak08,chen10a}. A significant quantity of gas is probed by
{\MgII} absorption, roughly 15\% of the total gas probed by damp Lyman
alpha absorbers, or 5\% of the total hydrogen in stars
\citep{kacprzak11c,menard12}.

A large body of evidence now suggests that {\MgII} absorption traces
both outflows from star forming galaxies
\citep{bouche06,tremonti07,zibetti07,martin09,weiner09,chelouche10,nestor11,noterdaeme10,bordoloi11,coil11,rubin10,menard12}
and accretion onto host galaxies
\citep{steidel02,chen10a,chen10b,kacprzak10a,kacprzak11b,stewart11b,ribaudo11,kacprzak12,rubin12}.
While it is clear that both processes are occurring, it is difficult
to disentangle which absorption systems are associated with each
process.  Comparing both the absorption-line and host galaxy
metallicity provides a suitable test for each scenario
\citep[e.g.,][and references therein]{ribaudo11,kacprzak12}, however
it is not yet feasible to perform this experiment on a large number of
objects.

Another promising avenue is exploring the relative {\MgII} absorption
and galaxy geometry/orientation. Modeling the morphologies of 40 {\it
HST} imaged absorbers, \citet{kacprzak11b} found that the {\MgII} gas
has co-planer geometry, but is not necessarily disk-like, that is
coupled to the galaxy inclination. This result provides evidence that
{\MgII} is probing accretion, consistent with the models of
\citet{stewart11b}, and not produced by star-burst driven winds.  By
stacking $\sim4000$ background galaxy spectra, \citet{bordoloi11}
determined statistically that {\MgII} absorption is stronger along the
host galaxy minor axis out to $\sim$50~kpc, suggesting winds eject gas
along the minor axis.  Using only 10 galaxies, \citet{bouche11} showed
that the galaxy azimuthal orientation with respect to quasar
sight-lines is bi-modal, indicating that the lines-of-sight probes
both winds and accretion. This is further validated by
\citet{churchill12a} who showed that weaker {\MgII} absorption likely
probes filamentary structures, intermediate strength {\MgII}
absorption likely probes co-planer accreting material, and the
strongest absorption primarily probes thick disks and out-of-plane
wind related and/or tidally disrupted material.

In this {\it Letter}, we explore the azimuthal distribution of all
published {\MgII} absorbers ($\geq$0.1~\AA) and non-absorbers
($<$0.1~\AA) around their spectroscopically confirmed host galaxies.
We compute the cumulative azimuthal angle probability distribution
function and show that it is bi-model for absorbers and flat for
non-absorbers. We further study this distribution in terms of galaxy
colors and equivalent widths. Throughout we adopt a $h=0.70$,
$\Omega_{\rm M}=0.3$, $\Omega_{\Lambda}=0.7$ cosmology.

%Similarly, in stacked spectra of thousands of local star-
%forming galaxies from SDSS/DR7, Chen et al. (2010c) showed that the
%blue-shifted NaD I absorption is stronger within 60deg of the mi- nor
%axis.


\section{The Sample}


\section{Galaxy Masses and Virial Radii}

The galaxy virial masses were obtained by performing halo abundance
matching
\citep[HAM,][]{kravtsov04,tasitsiomi04,vale04,conroy06,conroy09,guo10,behroozi10,firmani10,trujillo-gomez11,rodriguez-puebla12}
%(HAM; Kravtsov et al. 2004, Tasitsiomi et al. 2004, Vale \&
%Ostriker 2004, Conroy et al. 2006, Conroy \& Wechsler 2009, Guo et
%al. 2010, Behroozi et al. 2010, Firmani et al. 2010, Trujillo-Gomez et
%al. 2011, Rodriguez-Puebla et al. 2012) 
of the dark matter (DM) halos in the Bolshoi $N$-body cosmological
simulation (Klypin et al.  2011) to the luminosity function calculated
by Wolf et al. (2003) from the COMBO-17 survey as a function of
redshift. In short, halo abundance matching links the luminosity of
galaxies to a property of the dark matter halos (such as the mass or
circular velocity) in a monotonic fashion which reproduces the
luminosity function by construction.

Following Trujillo-Gomez et al. 2011, we choose the
maximum circular velocity ( $V_{\rm circ} =
\sqrt{\frac{GM(<r)}{r}}\Big|_{\rm max}$ ) of the halos since it is a
direct probe of the depth of the potential well, and contrary to the
virial mass, it is unambiguously defined for distinct halos as well as
for subhalos. We refer the reader to Trujillo-Gomez et al. 2011 for
more details on the procedure. The HAM model contains no free
parameters and has been extremely successful in reproducing many
galaxy statistics such as the two-point correlation function as a
function of redshift (Conroy et al. 2006) and luminosity
(Trujillo-Gomez et al. 2011), the luminosity-velocity relation, the
baryonic Tully-Fisher relation, as well as the galaxy velocity
function (Trujillo-Gomez et al. 2011). In addition, halo abundance
matching yields galaxy baryon fractions that agree with direct
estimates from lensing and satellite kinematics within the
uncertainties in the observations (e.g. Dutton et. al 2011).

The HAM was applied to the Bolshoi dark matter halo catalogs for the
redshifts corresponding to the five bins (centered at
z=0.3,0.5,0.7,0.9,1.1, with dz=0.2) used in the calculation of the
COMBO-17 luminosity function in the r-band. Since some of the galaxies
in our sample have redshifts z < 0.2, we extended the LF in the bin
0.2 < z < 0.4 down to z=0, assuming that the luminosity function does
not evolve significantly from z=0.3 to z=0. This avoids the
complications of using the ``local'' SDSS luminosity function which
seems to be incompatible at the bright end with COMBO-17 possibly due
to the different sensitivity of the two surveys (Wolf et al. 2003).

To obtain a virial mass of each galaxy in the sample, once the HAM
assigns an r-band luminosity to each DM halo in the catalogs, we
calculated the average mass of all the halos that fall in a bin with
$\delta M_{\rm r} = 0.1$ of the measured value for that galaxy. Since
halo abundance matching is a statistical method, each derived mass
should be interpreted as the average virial mass of a halo which hosts
a galaxy with a given luminosity. The intrinsic scatter between virial
mass and r-band luminosity originates from the scatter in the mass
vs. circular velocity relation due to the variation in formation times
of halos of the same mass. This scatter is taken into account and
shown as the 1-sigma error bars in the virial masses in Figure
(?). Figure (?) compares the average mass and the scatter calculated
with two different bin sizes ($\delta M_r = 0.1$ and $\delta M_r =
0.4$). The luminosity bin size has has virtually no effect on the
scatter of each mass estimate.

Since the abundance of DM halos is known very precisely in the
concordance cosmology, the main source of uncertainty in the derived
masses is the observed luminosity function. To test the range of
possible systematics in our mass estimates we computed the variation
in the obtained galaxy virial masses assuming that the observed LF
evolution is dominated by systematic measurement errors. Figure (?)
shows the systematic error regions in the luminosity-mass relation
obtained at two different epochs. Even with this extreme assumption
about the systematic errors, our results are qualitatively unchanged.

We plot Mvir vs. Mr to the lowest halo mass that is resolved in the
simulations.  This minimum mass is dictated by the completeness of the
velocity function (to maximum circular velocity $v>50$ {\kms}) as
discussed in \citet{trujillo-gomez11} and Klypin et al.  2011.  The
truncation is at brighter luminosity at higher redshift because of the
steeper luminosity function at high redshift, i.e., the smallest halo
mass gets assigned to a brighter galaxy.

%%%%%%%%%%%%%%%%%%%%%%%%%%%%%%%%%%%%%%%%%%%%%%%%%%%%%%%%%%%%%%%%%%
\begin{figure}
\epsscale{1.1}
\plotone{../Data/MvsW.eps}
\caption[angle=0]{The galaxy color dependence of the Azimuthal
distribution of {\MgII}. The solid line (black) is the bi-modal
distribution of {\MgII} absorption shown in Figure~\ref{fig:main}. The
color selection was adopted from \citet{chen10a} of $B_{AB}-R_{AB}
\leq 1.1$ represent late-type galaxies (dashed blue line) and
$B_{AB}-R_{AB} > 1.1$ are early-type galaxies (dotted red
line). Star-forming spirals are dominated by outflows. }
\label{fig:Phi-masses}
\end{figure}
%%%%%%%%%%%%%%%%%%%%%%%%%%%%%%%%%%%%%%%%%%%%%%%%%%%%%%%%%%%%%%%%%%


%%%%%%%%%%%%%%%%%%%%%%%%%%%%%%%%%%%%%%%%%%%%%%%%%%%%%%%%%%%%%%%%%%
\begin{figure}
\epsscale{1.1}
\plotone{../Data/MvsW-us.eps}
\caption[angle=0]{The galaxy color dependence of the Azimuthal
distribution of {\MgII}. The solid line (black) is the bi-modal
distribution of {\MgII} absorption shown in Figure~\ref{fig:main}. The
color selection was adopted from \citet{chen10a} of $B_{AB}-R_{AB}
\leq 1.1$ represent late-type galaxies (dashed blue line) and
$B_{AB}-R_{AB} > 1.1$ are early-type galaxies (dotted red
line). Star-forming spirals are dominated by outflows. }
\label{fig:Phi-masses}
\end{figure}
%%%%%%%%%%%%%%%%%%%%%%%%%%%%%%%%%%%%%%%%%%%%%%%%%%%%%%%%%%%%%%%%%%


\section{Conclusion}

%%%%%%%%%%%%%%%%%%%%%%%%%%%%%%%%%%%%%%%%
\acknowledgments 

CWC and NMN were supported through grant HST-GO-11667.01-A provided by
NASA via the Space Telescope Science Institute, which is operated by
the Association of Universities for Research in Astronomy (AURA) under
NASA contract NAS 5-26555.  Some observations were obtained at the
W.M. Keck Observatory, which is operated as a scientific partnership
among the California Institute of Technology, the University of
California and NASA.  The Observatory was made possible by the
generous financial support of the W.M. Keck Foundation.  Data was also
from The Sloan Digital Sky Survey (SDSS).  Funding for the SDSS and
SDSS-II has been provided by the Alfred P. Sloan Foundation, the
Participating Institutions, the National Science Foundation, the
U.S. Department of Energy, the National Aeronautics and Space
Administration, the Japanese Monbukagakusho, the Max Planck Society,
and the Higher Education Funding Council for England. The SDSS Web
Site is http://www.sdss.org/. The SDSS is managed by the Astrophysical
Research Consortium for the Participating Institutions.


{\it Facilities:} \facility{HST (WFPC--2)}, \facility{Keck I (HIRES,
LRIS)}, \facility{VLT (UVES)}, \facility{Sloan (SDSS)}.


\appendix

This is the appendix.
This is the appendix.
This is the appendix.
This is the appendix.
This is the appendix.
This is the appendix.
This is the appendix.
This is the appendix.
This is the appendix.

%%%%%%%%%%%%%%%%%%%%%%%%%%%%%%%%%%%%%%%%%%%%%%%%%%%%%%%%%%%%%%%%%%
\begin{figure*}
\epsscale{1.1}
\plotone{../Data/Test-Actual/results/actual.eps}
\caption[angle=0]{The galaxy color dependence of the Azimuthal
distribution of {\MgII}. The solid line (black) is the bi-modal
distribution of {\MgII} absorption shown in Figure~\ref{fig:main}. The
color selection was adopted from \citet{chen10a} of $B_{AB}-R_{AB}
\leq 1.1$ represent late-type galaxies (dashed blue line) and
$B_{AB}-R_{AB} > 1.1$ are early-type galaxies (dotted red
line). Star-forming spirals are dominated by outflows. }
\label{fig:Phi-masses}
\end{figure*}
%%%%%%%%%%%%%%%%%%%%%%%%%%%%%%%%%%%%%%%%%%%%%%%%%%%%%%%%%%%%%%%%%%



\begin{thebibliography}{}

\bibitem[Behroozi, Conroy, \& Wechsler(2010)Behroozi{\etal}]{behroozi10} 
Behroozi, P.~S., Conroy, C., \& Wechsler, R.~H.\ 2010, \apj, 717, 379

\bibitem[Conroy {\etal}(2006)]{conroy06} Conroy, C., Wechsler, R.~H.,
\& Kravtsov, A.~V.\ 2006, \apj, 647, 201

\bibitem[Conroy \& Wechsler(2009)]{conroy09} Conroy, C., \&
Wechsler, R.~H.\ 2009, \apj, 696, 620

\bibitem[Guo {\etal}(2010)]{guo10} Guo, Q., White, S., Li, C., 
\& Boylan-Kolchin, M.\ 2010, \mnras, 404, 1111 

\bibitem[Firmani {\etal}(2010)]{firmani10} Firmani, C., Avila-Reese,
V., \& Rodr{\'{\i}}guez-Puebla, A.\ 2010, \mnras, 404, 1100

\bibitem[Kravtsov {\etal}(2004)]{kravtsov04} Kravtsov, A.~V., Berlind,
A.~A., Wechsler, R.~H., et al.\ 2004, \apj, 609, 35

\bibitem[Nielsen, Churchill, \& Kacprzak(2012)]{nielsen12}
Nielsen, N. M., Churchill, C.~W., \& Kacprzak, G.~G. \ 2012, \apjl, in prep

\bibitem[Rodriguez-Puebla {\etal}(2012)]{rodriguez-puebla12} 
Rodriguez-Puebla, A., Drory, N., \& Avila-Reese, V.\ 2012, arXiv:1204.0804 

\bibitem[Tasitsiomi {\etal}(2004)]{tasitsiomi04} Tasitsiomi, A.,
Kravtsov, A.~V., Wechsler, R.~H., \& Primack, J.~R.\ 2004, \apj, 614,
533

\bibitem[Trujillo-Gomez {\etal}(2011)]{trujillo-gomez11}
Trujillo-Gomez, S., Klypin, A., Primack, J., \& Romanowsky, A.~J.\
2011, \apj, 742, 16

\bibitem[Vale \& Ostriker(2004)]{vale04} Vale, A., \& Ostriker, J.~P.\
2004, \mnras, 353, 189


%\bibitem[Adelberger {\etal}(2003)]{adelberger03} 
%Adelberger, K.~L., Steidel, C.~C., Shapley, A.~E., \& Pettini, M. 2003, ApJ, 584, 45 

%\bibitem[Asplund {\etal}(2009)]{asplund09} Asplund, M., Grevesse, N.,
%Sauval, A.~J., \& Scott, P.\ 2009, \araa, 47, 481

%\bibitem[Bahcall et al.(1993)]{bachall93} 
%Bahcall, J.~N., Bergeron, J., Boksenberg, A., et al.\ 1993, \apjs, 87, 1

%\bibitem[Bahcall et al.(1996)]{bahcall96} 
%Bahcall, J.~N., Bergeron, J., Boksenberg, A., et al.\ 1996, \apj, 457, 19

%\bibitem[Barlow(2003)]{barlow03} 
%Barlow, R.\ 2003, arXiv:physics/0306138

%\bibitem[Barton \& Cooke(2009)]{barton09} 
%Barton, E.~J., \& Cooke, J.\ 2009, AJ, 138, 1817

%\bibitem[Bechtold et al.(2001)]{bechtold01} 
%Bechtold, J., Siemiginowska, A., Aldcroft, T.~L., Elvis, M., \&
%Dobrzycki, A.\ 2001, ApJ, 562, 133

%\bibitem[Bergeron(1986)]{bergeron86} 
%Bergeron, J.\ 1986, A\&A, 155, L8 

%\bibitem[Bergeron(1988)]{bergeron88} Bergeron, J.\ 1988, Large 
%Scale Structures of the Universe, 130, 343 

%\bibitem[Bergeron \& Boiss\'{e}(1991)]{bb91}
%Bergeron, J., \& Boiss\'{e}, P. 1991, A\&A, 243, 334

%\bibitem[Bergeron, Cristiani, \& Shaver(1992)]{bergeron92}
%Bergeron, J., Cristiani, S., \& Shaver, P. A. 1992, A\&A, 257, 417

%\bibitem[Bergeron \&  Stas\'{i}nska(1986)]{bergeron86}
%Bergeron, J. \& Stas\'{i}nska, G. 1986, A\&A, 169, 1

%\bibitem[Bond {\etal}(2001a)]{bond01a}
%Bond, N. A., Churchill, C. W., Charlton, J. C., \& Vogt, S. S. 2001, ApJ, 557, 761

%\bibitem[Bond {\etal}(2001)]{bond01}
%Bond, N. A., Churchill, C. W., Charlton, J. C., \& Vogt, S. S. 2001, ApJ, 562, 641

%\bibitem[Bechtold \& Ellingson(1992)]{bechtold92} 
%Bechtold, J., \& Ellingson, E.\ 1992, ApJ, 396, 20

%\bibitem[Bertin \& Arnouts(1996)]{bertin96}
%Bertin, E., \& Arnouts, S. 1996, A\&AS, 117, 393

%\bibitem[Bordoloi {\etal}(2011)]{bordoloi11} 
%Bordoloi, R., Lilly, S.~J., Knobel, C., {\etal} 2011, arXiv:1106.0616

%\bibitem[Bouch\'{e} {\etal}(2011)]{bouche11}
%Bouch\'{e}, N., Hohensee, W., Vargas, R., Kacprzak, G. G., Martin,
%C. L., Cooke, J., \& Churchill, C. W. 2011, arXiv1110.5877

%\bibitem[Bouch\'{e} {\etal}(2006)]{bouche06}
%Bouch\'{e}, N., Murphy, M. T., P\'{e}roux, C., Csabai, I. \& Wild. V. 2006 MNRAS, 371, 495

%\bibitem[Bouch{\'e} et al.(2007)]{bouche07} 
%Bouch{\'e}, N., Murphy, M.~T., P{\'e}roux, C., Davies, R., Eisenhauer,
%F., F{\"o}rster Schreiber, N.~M., \& Tacconi, L.\ 2007, ApJL, 669, L5

%\bibitem[Bowen et al.(2005)]{bowen05} 
%Bowen, D.~V., Jenkins, E.~B., Pettini, M., \& Tripp, T.~M.\ 2005,
%ApJ, 635, 880

%\bibitem[Brooks et al.(2009)]{brooks09} 
%Brooks, A.~M., Governato, F., Quinn, T., Brook, C.~B., \& Wadsley, J.\
%2009, \apj, 694, 396

%\bibitem[Bruzual \& Charlot(2003)]{bruzual03} 
%Bruzual, G., \& Charlot, S.\ 2003, \mnras, 344, 1000

%\bibitem[Bryan \& Norman(1998)]{bryan98} 
%Bryan, G.~L., \& Norman, M.~L.\ 1998, \apj, 495, 80

%\bibitem[Burkert \& Lin(2000)]{burkert00}
%Burkert, A., \& Lin, D.~N.~C.\ 2000, ApJ, 537, 270

%\bibitem[Carilli et al.(1996)]{carilli96} 
%Carilli, C.~L., Lane, W., de Bruyn, A.~G., Braun, R., \& Miley, G.~K.\
%1996, AJ, 111, 1830

%\bibitem[Ceverino et al.(2010)]{ceverino10} 
%Ceverino, D., Dekel, A., \& Bournaud, F.\ 2010, \mnras, 404, 2151

%\bibitem[Chabrier(2003)]{chabrier03} 
%Chabrier, G.\ 2003, \pasp, 115, 763

%\bibitem[Charlton \& Churchill(1996)]{cc96}
%Charlton, J.~C., \& Churchill, C.~W. 1996, ApJ, 465, 631

%\bibitem[Chelouche \& Bowen(2010)]{chelouche10}
%Chelouche, D., \& Bowen, D.~V.\ 2010, \apj, 722, 1821 

%\bibitem[Chelouche et al.(2008)]{chelouche08} 
%Chelouche, D., M{\'e}nard, B., Bowen, D.~V., \& Gnat, O.\ 2008, ApJ,
%683, 55

%\bibitem[Chen {\etal}(2010)]{chen10a} 
%Chen, H.-W., Helsby, J.~E., Gauthier, J.-R., Shectman, S.~A.,
%Thompson, I.~B., \& Tinker, J.~L.\ 2010a, \apj, 714, 1521

%\bibitem[Chen et al.(2005)]{chen05} 
%Chen, H.-W., Kennicutt, R.~C., Jr., \& Rauch, M.\ 2005, ApJ, 620, 703

%\bibitem[Chen \& Lanzetta(2003)]{chen03}
%Chen, H.-W. \& Lanzetta, K. M. 2003, ApJ, 597, 706

%\bibitem[Chen et al.(2001)]{chen01} 
%Chen, H.-W., Lanzetta, K.~M., \& Webb, J.~K.\ 2001, \apj, 556, 158

%\bibitem[Chen \& Tinker(2008)]{chen08} 
%Chen, H.-W., \& Tinker, J.~L.\ 2008, ApJ, 687, 745 

%\bibitem[Chen et al.(2010)]{chen10b} 
%Chen, H.-W., Wild, V., Tinker, J.~L., et al.\ 2010, \apjl, 724, L176

%\bibitem[Chengalur \& Kanekar(2000)]{chengalur00} 
%Chengalur, J.~N., \& Kanekar, N.\ 2000, MNRAS, 318, 303

%\bibitem[Chun {\etal}(2006)]{chun06} Chun, M.~R., Gharanfoli, 
%S., Kulkarni, V.~P., \& Takamiya, M.\ 2006, AJ, 131, 686 

%\bibitem[Churchill(1997)]{cwcthesis} 
%Churchill, C. W. 1997, Ph.D. Thesis, University of California, Santa
%Cruz

%\bibitem[Churchill \& Le~Brun(1998)]{cl98}
%Churchill, C. W., \& Le Brun, V. 1998, ApJ, 499, 677

%\bibitem[Churchill(2012c)]{churchill12c} 
%Churchill, C. W. 2012, apJ, in prep

%\bibitem[Churchill {\etal}(2012b)]{churchill12b} 1317
%Churchill et al. to be submitted.

%\bibitem[Churchill, Kacprzak, \& Steidel(2005)]{cwc-china}
%Churchill, C. W., Kacprzak, G. G., \& Steidel, C. C. 2005, in {\it
%Probing Galaxies through Quasar Absorption Lines}, IAU 199
%Proceedings, eds.\ P. R. Williams, C.--G. Shu, \& B. M\'{e}nard
%(Cambridge: Cambridge University Press), p.\ 24

%\bibitem[Churchill {\etal}(2012a)]{churchill12a}
%Churchill, C.~W., Kacprzak, G.~G., Nielsen, N. M., Steidel, C.~C., \&
%Murphy, M.~T.\ 2012, \apj, submitted

%\bibitem[Churchill {\etal}(2007)]{churchill07}
%Churchill, C. W., Kacprzak, G. G., Steidel, C. C. \& Evans, J. L. 2007, ApJ, 661, 714

%\bibitem[Churchill {\etal}(2000a)]{archiveI}
%Churchill, C. W., Mellon, R. R., Charlton, J. C., Jannuzi, B. T.,
%Kirhakos, S., Steidel, C. C., \& Schneider, D. P. 2000a, ApJS, 130, 91

%\bibitem[Churchill {\etal}(2000b)]{archiveII}
%Churchill, C. W., Mellon, R. R., Charlton, J. C., Jannuzi, B. T.,
%Kirhakos, S., Steidel, C. C., \& Schneider, D. P. 2000b, ApJ, 543, 577

%\bibitem[Churchill {\etal}(1999)]{weakI}
%Churchill, C. W., Rigby, J. R., Charlton, J. C., \& Vogt, S. S. 1999, ApJS, 120, 51

%\bibitem[Churchill \& Steidel(2003)]{cs03} 
%Churchill, C., \& Steidel, C.\ 2003, The IGM/Galaxy Connection.~The
%Distribution of Baryons at z=0, 281, 149

%\bibitem[Churchill, Steidel, \& Vogt(1996)]{csv96}
%Churchill, C. W., Steidel, C. C., \& Vogt, S. S. 1996, ApJ, 471, 164

%\bibitem[Churchill \& Vogt(2001)]{cv01}
%Churchill, C. W., \& Vogt, S. S. 2001, AJ, 122, 679 

%\bibitem[Churchill, Vogt, \& Charlton(2001)]{cvc03}
%Churchill, C. W., Vogt, S. S., \& Charlton, J. C. 2003, AJ, 125, 98 

%\bibitem[Coil et al.(2011)]{coil11} 
%Coil, A.~L., Weiner, B.~J., Holz, D.~E., et al.\ 2011, \apj, 743, 46

%\bibitem[Cooksey et al.(2008)]{cooksey08}
% Cooksey, K.~L., Prochaska, J.~X., Chen, H.-W., Mulchaey, J.~S., \&
%Weiner, B.~J.\ 2008, \apj, 676, 262

%\bibitem[Curran et al.(2005)]{curran05} 
%Curran, S.~J., Murphy, M.~T., Pihlstr{\"o}m, Y.~M., Webb, J.~K., \&
%Purcell, C.~R.\ 2005, MNRAS, 356, 1509

%\bibitem[Chynoweth et al.(2008)]{chynoweth08} 
%Chynoweth, K.~M., Langston, G.~I., Yun, M.~S., Lockman, F.~J., Rubin,
%K.~H.~R., \& Scoles, S.~A.\ 2008, AJ, 135, 1983

%\bibitem[D'Agonstini(2000)]{d'agostini00a} 
%D'Agonstini, G.\ 2000, arXiv:physics/0403086

%\bibitem[D'Agonstini \& Raso(2000)]{d'agostini00b} 
%D'Agonstini, G., \& Raso, M.\ 2000, arXiv:physics/0002056

%\bibitem[Dav{\'e} et al.(1999)]{dave99} 
%Dav{\'e}, R., Hernquist, L., Katz, N., \& Weinberg, D.~H.\ 1999, \apj,
%511, 521

%\bibitem[Dekel \& Birnboim(2006)]{dekel06} 
%Dekel, A., \& Birnboim, Y.\ 2006, \mnras, 368, 2

%\bibitem[Dekel et al.(2009)]{dekel09} 
%Dekel, A., Birnboim, Y., Engel, G., et al.\ 2009, \nat, 457, 451

%\bibitem[Dekker {\etal}(2000)]{dekker00} 
%Dekker, H., D'Odorico, S., Kaufer, A. Delabre, B. \& Kotzlowski H. 2000, SPIE, 4008, 534

%\bibitem[Ding, Charlton, \& Churchill(2005)]{ding05}
%Ding, J., Charlton, J. C., \& Churchill, C. W. 2005, ApJ, 621, 615

%\bibitem[Dixon {\etal}(2010)]{cos-ihb}
%Dixon, W. V., {\etal} 2010, Cosmic Origins Spectrograph Instrument
%Handbook, Version 3.0 (Baltimore: STScI)

%\bibitem[Draine(2011)]{draine11} Draine, B.~T.\ 2011, Physics of the
%Interstellar and Intergalactic Medium, Princeton University Press,
%ISBN: 978-0-691-12214-4 (Table 1.4, p8)

%\bibitem[Ellison {\etal}(2003)]{ellison03} 
%Ellison, S.~L., Mall{\'e}n-Ornelas, G., \& Sawicki, M.\ 2003, ApJ,
% 589, 709

%\bibitem[Erb et al.(2006)]{erb06} 
%Erb, D.~K., Steidel, C.~C., Shapley, A.~E., Pettini, M., Reddy, N.~A.,
%\& Adelberger, K.~L.\ 2006, ApJ, 646, 107

%\bibitem[Faber {\etal}(2007)]{faber07} 
%Faber, S.~M., et al.\ 2007, ApJ, 665, 265

%\bibitem[Faucher-Gigu{\`e}re et al.(2011)]{faucher-giguere11} 
%Faucher-Gigu{\`e}re, C.-A., Kere{\v s}, D., 
%\& Ma, C.-P.\ 2011, \mnras, 417, 2982 

%\bibitem[Ferland et al.(1998)]{ferland98} 
%Ferland, G.~J., Korista, K.~T., Verner, D.~A., et al.\ 1998, \pasp, 110, 761

%\bibitem[Woods {\etal}(2006)]{woods06} 
%Woods, D.~F., Geller, M.~J., \& Barton, E.~J.\ 2006, AJ, 132, 197 

%\bibitem[Fraternali {\etal}(2002)]{fraternali02}
%Fraternali, F., van Moorsel, G., Sancisi, R., \& Oosterloo, T. 2002,
%AJ, 123, 312

%\bibitem[Fukugita, Shimasaku, \& Ichikawa(1995)]{fukugita95}
%Fukugita, M., Shimasaku, K., \& Ichikawa, T. 1995, PASP, 107, 945 

%\bibitem[Fumagalli et al.(2011)]{fumagalli11} Fumagalli, M., 
%Prochaska, J.~X., Kasen, D., et al.\ 2011, \mnras, 418, 1796 

%\bibitem[Guillemin \& Bergeron(1997)]{gb97}
%Guillemin p., \& Bergeron, J. 1997, A\&A, 328, 499

%\bibitem[Haehnelt et al.(1998)]{haehnelt98} 
%Haehnelt, M.~G., Steinmetz, M., \& Rauch, M.\ 1998, ApJ, 495, 647

%\bibitem[Haardt \& Madau(2011)]{haardt11} Haardt, F., \& Madau, P.\
%2011, ApJ, arXiv:1105.2039

%\bibitem[Herbert-Fort et al.(2006)]{herbert-fort06} 
%Herbert-Fort, S., Prochaska, J.~X., Dessauges-Zavadsky, M., Ellison,
%S.~L., Howk, J.~C., Wolfe, A.~M., \& Prochter, G.~E.\ 2006, \pasp,
%118, 1077

%\bibitem[Hewitt \& Burbidge(1993)]{hb93}
%Hewitt, A., \& Burbidge, G. 1993, ApJS, 87, 451 (HB93)

%\bibitem[Heckman(2002)]{heckman02} 
%Heckman, T.~M.\ 2002, Extragalactic Gas at Low Redshift, 254, 292

%\bibitem[Heckman(2003)]{heckman03} 
%Heckman, T.~M.\ 2003, Revista Mexicana de Astronomia y Astrofisica
%Conference Series, 17, 47

%\bibitem[Holweger(2001)]{holweger01} 
%Holweger, H.\ 2001, Joint SOHO/ACE workshop ''Solar and Galactic
%Composition'', 598, 23

%\bibitem[Jannuzi et al.(1998)]{jannuzi98} 
%Jannuzi, B.~T., Bahcall, J.~N., Bergeron, J., et al.\ 1998, \apjs, 118, 1

%\bibitem[Jenkins {\etal}(2005)]{jenkins05}
%Jenkins, E.B., Bowen, D.V., Tripp, T.M., \& Sembach, K.R. 2005, ApJ, 623, 767

%\bibitem[Kacprzak \& Churchill(2011)]{kacprzak11c} 
%Kacprzak, G.~G., \& Churchill, C.~W.\ 2011c, \apjl, 743, L34

%\bibitem[Kacprzak {\etal}(2011a)]{kacprzak11a} 
% Kacprzak, G.~G., Churchill, C.~W., Barton, E.~J., \& Cooke, J.\ 2011a,
%\apj, 733, 105

%\bibitem[Kacprzak {\etal}(2010a)]{kacprzak10a} 
%Kacprzak, G.~G., Churchill, C.~W., Ceverino, D., Steidel, C.~C.,
%Klypin, A., \& Murphy, M.~T.\ 2010a, ApJ, 711, 533

%\bibitem[Kacprzak {\etal}(2011b)]{kacprzak11b} 
%Kacprzak, G.~G., Churchill, C.~W., Evans, J.~L., Murphy, M.~T., \&
%Steidel, C.~C.\ 2011b, \mnras, 416, 3118 

%\bibitem[Kacprzak et al.(2010b)]{kacprzak10b} 
%Kacprzak, G.~G., Murphy, M.~T., \& Churchill, C.~W.\ 2010, \mnras, 406, 445

%\bibitem[Kacprzak {\etal}(2008)]{kacprzak08} 
%Kacprzak, G.~G., Churchill, C.~W., Steidel, C.~C., \& Murphy, M.~T.\
%2008, AJ, 135, 922

%\bibitem[Kacprzak {\etal}(2007)]{kacprzak07} 
%Kacprzak, G. G., Churchill, C. W., Steidel, C. C., Murphy, M. T., \& Evans, J. L 2007, ApJ, 662, 909 

%\bibitem[Kacprzak {\etal}(2012)]{kacprzak12} 
%Kacprzak, G. G., Churchill, C. W., Steidel, C. C., Spitler, L. R., \&
%Holtzman, J. A. 2012, MNRAS, submitted

%\bibitem[Kanekar \& Chengalur(2001b)]{kanekar01b}
% Kanekar, N., \& Chengalur, J.~N.\ 2001b, A\&A, 369, 42

%\bibitem[Kanekar \& Chengalur(2001a)]{kanekar01} 
%Kanekar, N., \& Chengalur, J.~N.\ 2001a, MNRAS, 325, 631

%\bibitem[Kanekar et al.(2009)]{kanekar09} 
%Kanekar, N., Smette, A., Briggs, F.~H., \& Chengalur, J.~N.\ 2009,
%ApJl, 705, L40

%\bibitem[Kato, Omachi, \& Aso(2002)]{kato02}
%Kato, T., Omachi, S., \& Aso, H. 2002, Lect. Notes Comput. Sci. 2396, 405

%\bibitem[Kaufmann et al.(2009)]{kaufmann09} Kaufmann, T., Bullock, 
%J.~S., Maller, A.~H., Fang, T., \& Wadsley, J.\ 2009, MNRAS, 396, 191 

%\bibitem[Kere{\v s} et al.(2009)]{keres09} 
%Kere{\v s}, D., Katz, N., Fardal, M., Dav{\'e}, R., \& Weinberg,
%D.~H.\ 2009, \mnras, 395, 160

%\bibitem[Kere{\v s} et al.(2005)]{keres05} 
%Kere{\v s}, D., Katz, N., Weinberg, D.~H., \& Dav{\'e}, R.\ 2005,
%\mnras, 363, 2

%\bibitem[Kewley et al.(2004)]{kewley04} 
%Kewley, L.~J., Geller, M.~J., \& Jansen, R.~A.\ 2004, AJ, 127, 2002

%\bibitem[Kewley et al.(2002)]{kewley02} 
%Kewley, L.~J., Geller, M.~J., Jansen, R.~A., \& Dopita, M.~A.\ 2002,
%AJ, 124, 3135

%\bibitem[Kim, Goobar, \& Perlmutter(1996)]{kim96}
%Kim, A., Goobar, A., \& Perlmutter, S. 1996, PASP, 108, 190

%\bibitem[Kinney {\etal}(1996)]{kinney96}
%Kinney, A. L., Calzetti, D., Bohlin, R. C., McQuade, K.,
%Storchi--Bergmann, T.  \& Schmitt, H. R. 1996, ApJ, 467, 38

%\bibitem[Knierman et al.(2003)]{knierman03} 
%Knierman, K.~A., Gallagher, S.~C., Charlton, J.~C., Hunsberger, S.~D.,
%Whitmore, B., Kundu, A., Hibbard, J.~E., \& Zaritsky, D.\ 2003, AJ,
%126, 1227

%\bibitem[Kobulnicky \& Phillips(2003)]{kobulnicky03}
%Kobulnicky, H.~A., \& Phillips, A.~C.\ 2003, ApJ, 599, 1031

%\bibitem[Kriss(2011)]{kriss11} 
%Kriss, G. A., COS Instrument Handbook 2011-01, (Baltimore, STScI)

%\bibitem[Kulkarni et al.(2005)]{kulkarni05} 
%Kulkarni, V.~P., Fall, S.~M., Lauroesch, J.~T., York, D.~G., Welty,
%D.~E., Khare, P., \& Truran, J.~W.\ 2005, ApJ, 618, 68

%\bibitem[Lacy et al.(2003)]{lacy03} 
%Lacy, M., Becker, R.~H., Storrie-Lombardi, L.~J., Gregg, M.~D.,
%Urrutia, T., \& White, R.~L.\ 2003, AJ, 126, 2230

%\bibitem[Lane {\etal}(1998)]{lane98} 
%Lane, W., Smette, A., Briggs, F., Rao, S., Turnshek, D., \& Meylan,
%G.\ 1998, AJ, 116, 26

%\bibitem[Lanzetta \& Bowen(1990)]{lanzetta90}
%Lanzetta, K.~M., \& Bowen, D.\ 1990, ApJ, 357, 321 

%\bibitem[Lanzetta \& Bowen(1992)]{lanzetta92}   
%Lanzetta, K. M. \& Bowen, D. V. 1992, ApJ, 391, 48L

%\bibitem[Lanzetta {\etal}(1995)]{lanzetta95} 
%Lanzetta, K.~M., Bowen, D.~V., Tytler, D., \& Webb, J.~K. 1995, ApJ,
%442, 538

%\bibitem[Le Brun {\etal}(1993)]{lebrun93} 
%Le Brun, V., Bergeron, J., Boisse, P., \& Christian, C.\ 1993, A\&A,
%279, 33

%\bibitem[Lin \& Murray(2000)]{lin00}
%Lin, D.~N.~C., \& Murray, S.~D.\ 2000, ApJ, 540, 170

%\bibitem[Lin \& Zou(2001)]{lin01} 
%Lin, W.-P., \& Zou, Z.-L.\ 2001, ChJAA, 1, 21

%\bibitem[Lopez et al.(2008)]{lopez08} 
%Lopez, S., et al.\ 2008, ApJ, 679, 1144

%\bibitem[Lynds \& Sandage(1963)]{lynds63}
%Lynds, C. R. \& Sandage, A. R. 1963, ApJ, 137, 1005

%\bibitem[Maller \& Bullock(2004)]{maller04}
%Maller, A.~H., \& Bullock, J.~S.\ 2004, MNRAS, 355, 694

%\bibitem[Mannucci et al.(2001)]{mannucci01} 
%Mannucci, F., Basile, F., Poggianti, B.~M., et al.\ 2001, \mnras, 326, 745

%\bibitem[Martin \& Bouch{\'e}(2009)]{martin09} 
%Martin, C.~L., \& Bouch{\'e}, N.\ 2009, \apj, 703, 1394 

%\bibitem[Matteucci et al.(1997)]{matteucci97} 
%Matteucci, F., Molaro, P., \& Vladilo, G.\ 1997, A\&A, 321, 45 

%\bibitem[McGaugh(2005)]{mcgaugh05}
% McGaugh, S.~S.\ 2005, \apj, 632, 859

%\bibitem[M{\'e}nard \& Chelouche(2009)]{menard09b}
%M{\'e}nard, B., \& Chelouche, D.\ 2009, \mnras, 393, 808

%\bibitem[M{\'e}nard \& Fukugita(2012)]{menard12}
%M{\'e}nard, B., \& Fukugita, M.\ 2012, arXiv:1204.1978

%\bibitem[Mo \& Miralda-Escude(1996)]{mo96}
%Mo, H.~J., \& Miralda-Escude, J.\ 1996, ApJ, 469, 589

%\bibitem[M{\o}ller et al.(2002)]{moller02} 
%M{\o}ller, P., Warren, S.~J., Fall, S.~M., Fynbo, J.~U., \& Jakobsen,
%P.\ 2002, ApJ, 574, 51

%\bibitem[Moster et al.(2010)]{moster10} 
%Moster, B.~P., Somerville, R.~S., Maulbetsch, C., et al.\ 2010, \apj,
%710, 903

%\bibitem[Mulchaey \& Zabludoff(1998)]{mulchaey98} 
%Mulchaey, J.~S., \& Zabludoff, A.~I.\ 1998, ApJ, 496, 73

%\bibitem[Masiero {\etal}(2005)]{joe05} 
%Masiero, J. R., Charlton, J. C., Ding, J., Churchill, C. W., \&
%Kacprzak, G. G. 2005, ApJ, 623, 57

%\bibitem[Milutinovi\'{c} {\etal}(2005)]{nikola05}
%Milutinovi\'{c}, N., Rigby, J. R., Masiero, J. R., Ryan, R. S., Palma,
%C., \& Charlton, J. C. 2005, ApJ, submitted, (astro--ph/0512238)

%\bibitem[Mo \& Mao(2002)]{mo02}
%Mo, H.~J., \& Mao, S. 2002, MNRAS, 333, 768 

%\bibitem[Mo \& Miralda-Escude(1996)]{mo96} 
%Mo, H.~J., \& Miralda-Escude, J.\ 1996, ApJ, 469, 589

%\bibitem[Monet {\etal}(1998)]{usno2}
%Monet, D., {\etal} 1998, USNO--SA2.0: A Catalog of Astrometric
%Standards (Washington: US Nav. Obs.)

%\bibitem[Narayanan {\etal}(2005)]{anand05}
%Narayanan, A, Charlton, J. C., Masiero, J. R., \& Lynch, R. 2005, ApJ,
%632, 92

%\bibitem[Narayanan et al.(2007)]{narayanan07} 
%Narayanan, A., Misawa, T., Charlton, J.~C., \& Kim, T.-S.\ 2007, ApJ,
%660, 1093

%\bibitem[Narayanan et al.(2008)]{narayanan08} 
%Narayanan, A., Charlton, J.~C., Misawa, T., Green, R.~E., \& Kim,
%T.-S.\ 2008, ApJ, 689, 782

%\bibitem[Nestor {\etal}(2011)]{nestor11} 
%Nestor, D.~B., Johnson, B.~D., Wild, V., {\etal} 2011, \mnras, 412, 1559

%\bibitem[Nestor {\etal}(2005)]{nestor05} 
%Nestor, D.~B., Turnshek, D.~A., \& Rao, S.~M.\ 2005, ApJ, 628, 637 

%\bibitem[Nestor {\etal}(2002)]{nestor02} 
%Nestor, D.~B., Rao, S.~M., Turnshek, D.~A., Monier, E., Lane, W.~M.,
%\& Bergeron, J.\ 2002, Extragalactic Gas at Low Redshift, 254, 34

%\bibitem[Nestor et al.(2007)]{nestor07} 
%Nestor, D.~B., Turnshek, D.~A., Rao, S.~M., \& Quider, A.~M.\ 2007,
%ApJ, 658, 185

%\bibitem[Noterdaeme {\etal}(2010)]{noterdaeme10} 
%Noterdaeme, P., Srianand, R., \& Mohan, V.\ 2010, \mnras, 403, 906

%\bibitem[Ocvirk et al.(2008)]{ocvirk08} 
%Ocvirk, P., Pichon, C., \& Teyssier, R.\ 2008, \mnras, 390, 1326

%\bibitem[Oppenheimer et al.(2010)]{oppenheimer10} 
%Oppenheimer, B.~D., Dav{\'e}, R., Kere{\v s}, D., et al.\ 2010,
%\mnras, 406, 2325

%\bibitem[Padilla et al.(2009)]{padilla09} 
%Padilla, N., Lacerna, I., Lopez, S., Barrientos, L.~F., Lira, P.,
%Andrews, H., \& Tejos, N.\ 2009, MNRAS, 395, 1135

%\bibitem[P{\'e}roux et al.(2011)]{peroux11} 
%P{\'e}roux, C., Bouch{\'e}, N., Kulkarni, V.~P., York, D.~G., \&
%Vladilo, G.\ 2011, \mnras, 410, 2237

%\bibitem[Pettini \& Pagel(2004)]{pettini04} 
%Pettini, M., \& Pagel, B.~E.~J.\ 2004, MNRAS, 348, L59

%\bibitem[Pollack et al.(2009)]{pollack09} 
%Pollack, L.~K., Chen, H.-W., Prochaska, J.~X., \& Bloom, J.~S.\ 2009,
%ApJ, 701, 1605

%\bibitem[Prochaska(2006)]{prochaska06} 
%Prochaska, J.~X.\ 2006, \apj, 650, 272

%\bibitem[Prochaska et al.(2002)]{prochaska02} 
%Prochaska, J.~X., Ryan-Weber, E., \& Staveley-Smith, L.\ 2002, PASP,
%114, 1197

%\bibitem[Prochaska \& Wolfe(1998)]{prochaska98} 
%Prochaska, J.~X., \& Wolfe, A.~M.\ 1998, ApJ, 507, 113 

%\bibitem[Prochaska \& Wolfe(1997)]{prochaska97} 
%Prochaska, J.~X., \& Wolfe, A.~M.\ 1997, ApJ, 487, 73

%\bibitem[Puche {\etal}(1992)]{puche92}
%Puche, D., Westpfahl, D., Brinks, E., \& Roy, J. 1992, AJ, 103, 1841

%\bibitem[Rand(2000)]{rand00} 
%Rand, R. 2000, ApJ, 537, 13                 

%\bibitem[Rao et al.(2006)]{rao06} 
%Rao, S.~M., Turnshek, D.~A., \& Nestor, D.~B.\ 2006, ApJ, 636, 610

%\bibitem[Rao {\etal}(2003)]{rao03}
%Rao, S.~M., Nestor, D.~B., Turnshek, D.~A., Lane, W.~M., Monier,
%E.~M., \& Bergeron, J.\ 2003, ApJ, 595, 94

%\bibitem[Rao \& Turnshek(2000)]{rao00} 
%Rao, S.~M., \& Turnshek, D.~A.\ 2000, ApJS, 130, 1 

%\bibitem[Ribaudo et al.(2011)]{ribaudo11} 
%Ribaudo, J., Lehner, N., Howk, J.~C., et al.\ 2011, \apj, 743, 207

%\bibitem[Rigby, Charlton, \& Churchill(2002)]{weakII}
%Rigby, J. R., Charlton, J. C., \& Churchill, C. W. 2002, ApJ, 565, 743

%\bibitem[Rubin et al.(2012)]{rubin12} 
%Rubin, K.~H.~R., Prochaska, J.~X., Koo, D.~C., \& Phillips, A.~C.\
%2012, \apjl, 747, L26

%\bibitem[Rubin {\etal}(2010)]{rubin10}
% Rubin, K.~H.~R., Weiner, B.~J., Koo, D.~C., {\etal} 2010, \apj, 719, 1503

%\bibitem[Rubin {\etal}(2010)]{rubin10a} 
%Rubin, K.~H.~R., Prochaska, J.~X., Koo, D.~C., Phillips, A.~C., \&
%Weiner, B.~J.\ 2010, \apj, 712, 574

%\bibitem[Sancisi et al.(2008)]{sancisi08} 
%Sancisi, R., Fraternali, F., Oosterloo, T., \& van der Hulst, T.\ 2008, A\&A Rev., 15, 189 

%\bibitem[Savage \& Sembach(1991)]{savage91} 
%Savage, B.~D., \& Sembach, K.~R.\ 1991, \apj, 379, 245

%\bibitem[Savage, Tripp, \& Lu(1998)]{savage98}
%Savage, B. D., Tripp, T. M. \& Lu, L. 1998, AJ, 115, 436

%\bibitem[Savaglio et al.(2005)]{savaglio05} 
%Savaglio, S., Glazebrook, K., Le Borgne, D., et al.\ 2005, \apj, 635,
%260

%\bibitem[Schlegel et al.(1998)]{schlegel98} 
%Schlegel, D.~J., Finkbeiner, D.~P., \& Davis, M.\ 1998, \apj, 500, 525

%\bibitem[Schneider {\etal}(1993)]{schneider93}
%Schneider, D. P., {\etal} 1993, ApJS, 87, 45

%\bibitem[Sheinis {\etal}(2002)]{sheinis02}
%Sheinis, A. I., Bolte, M., Epps, H. W., Kibrick, R. I., Miller, J. S.,
%Radovan, M. V., Bigelow, B. C., \& Sutin, B. M. 2002, PASP. 114, 851

%\bibitem[Siemiginowska et al.(2002)]{siemiginowska02} 
%Siemiginowska, A., Bechtold, J., Aldcroft, T.~L., Elvis, M., Harris,
%D.~E., \& Dobrzycki, A.\ 2002, ApJ, 570, 543

%\bibitem[Siemiginowska et al.(2007)]{siemiginowska07} 
%Siemiginowska, A., Stawarz, {\L}., Cheung, C.~C., Harris, D.~E.,
%Sikora, M., Aldcroft, T.~L., \& Bechtold, J.\ 2007, ApJ, 657, 145

%\bibitem[Simard {\etal}(2002)]{simard02}
%Simard, L., Willmer, C. N. A., Vogt, N. P., Sarajedini, V. L.,
%Philips, A. C., Weiner, B. J., Koo, D. C., Im, M., Illingworth, G. D.,
%\& Faber, S. M. 2002, ApJS, 142, 1

%\bibitem[Skrutskie et al.(2006)]{skrutskie06} 
%Skrutskie, M.~F., Cutri, R.~M., Stiening, R., et al.\ 2006, \aj, 131,
%1163

%\bibitem[Spinrad {\etal}(1993)]{spinrad93}
%Spinrad, H., Filippenko, A. V., Yee, H. K., Ellingson, E., Blades, J. C.,
%Bahcall, J. N.; Jannuzi, B. T.; Bechtold, J., \& Dobrzycki, A. 1993 AJ,
%106, 1

%\bibitem[Spolaor {\etal}(2010)]{spolaor10} 
%Spolaor, M., Kobayashi, C., Forbes, D. A., Couch, W. J., \& Hau,
%G. T. 2010, MNRAS, submitted

%\bibitem[Steidel(1995)]{steidel95}
%Steidel, C. C. 1995, in QSO Absorption Lines, ed.\ G. Meylan, (Springer--verlag: Berlin
%Heidelberg), p.\ 139

%\bibitem[Steidel et al.(2010)]{steidel10} 
%Steidel, C.~C., Erb, D.~K., Shapley, A.~E., et al.\ 2010, \apj, 717,
%289

%\bibitem[Steidel {\etal}(2002)]{steidel02} 
%Steidel, C. C., Kollmeier, J. A., Shapely, A. E., Churchill, C. W.,
%Dickinson, M., \& Pettini, M. 2002, ApJ, 570, 526

%\bibitem[Steidel \& Dickinson(1992)]{sd92}
%Steidel, C. C., \& Dickinson, M.  ApJ, 394, 81

%\bibitem[Steidel {\etal}(1997)]{s97} 
%Steidel, C. C., Dickinson, M., Meyer, D. M., Adelberger, K. L., \&
%Sembach, K. R. 1997, ApJ, 480, 586

%\bibitem[Steidel, Dickinson, \& Persson(1994)]{sdp94}
%Steidel, C. C., Dickinson, M., \& Persson, S. E.  1994, ApJ, 437, L75

%\bibitem[Steidel \& Sargent(1992)]{ss92}
%Steidel, C. C., \& Sargent, W. L. W. 1992, ApJs, 80, 1

%\bibitem[Steidel {\etal}(2004)]{steidel04} 
%Steidel, C.~C., Shapley, A.~E., Pettini, M., Adelberger, K.~L., Erb, D.~K., Reddy, N.~A., \& Hunt, M.~P. 2004, ApJ, 604, 534

%\bibitem[Stetson(1989)]{stetson89}
%Stetson, P. B. 1989, Advanced School of Astrophysics, (Univerisidade
%de Sao Paulo), p.\ 1

%\bibitem[Stewart(2011c)]{stewart11c} 
%Stewart, K.~R.\ 2011, arXiv:1109.3207

%\bibitem[Stewart et al.(2009)]{stewart09} 
%Stewart, K.~R., Bullock, J.~S., Wechsler, R.~H., \& Maller, A.~H.\
%2009, \apj, 702, 307

%\bibitem[Stewart et al.(2011a)]{stewart11a} 
%Stewart, K.~R., Kaufmann, T., Bullock, J.~S., et al.\ 2011, \apjl,
%735, L1

%\bibitem[Stewart et al.(2011)]{stewart11b} 
%Stewart, K.~R., Kaufmann, T., Bullock, J.~S., et al.\ 2011, \apj, 738, 39

%\bibitem[Stocke {\etal}(2004)]{stocke04}
%Stocke, J. T., Keeney, B. A., McLin, K., Rosenberg, J. L., Weymann, R
%J., \& Giroux, M. L. 2004, ApJ, 609, 94

%\bibitem[Swaters, Sancisi \& van der Hulst(1997)]{swaters97}
%Swaters, R. A., Sancisi, R., \& van der Hulst, J. M. 1997, ApJ, 491, 140

%\bibitem[Thom et al.(2011)]{thom11} 
%Thom, C., Werk, J.~K., Tumlinson, J., et al.\ 2011, \apj, 736, 1

%\bibitem[Tinker \& Chen(2008)]{tinker08} 
%Tinker, J.~L., \& Chen, H.-W.\ 2008, ApJ, 679, 1218 

%\bibitem[Tremonti {\etal}(2007)]{tremonti07} 
%Tremonti, C.~A., Moustakas, J., \& Diamond-Stanic, A.~M.\ 2007, ApJL,
%663, L77

%\bibitem[Tripp \& Bowen(2005)]{tripp-china}
%Tripp, T. M., \& Bowen, D. V. 2005, in {\it Probing Galaxies through
%Quasar Absorption Lines}, IAU 199 Proceedings, eds.\ P. R. Williams,
%C.--G. Shu, \& B. M\'{e}nard (Cambridge: Cambridge University Press),
%p.\ 5

%\bibitem[Tripp et al.(2005)]{tripp05} 
%Tripp, T.~M., Jenkins, E.~B., Bowen, D.~V., et al.\ 2005, \apj, 619, 714

%\bibitem[Tumlinson et al.(2011)]{tumlinson11} 
%Tumlinson, J., Thom, C., Werk, J.~K., et al.\ 2011, arXiv:1111.3980

%\bibitem[Turnshek et al.(2003)]{turnshek03} 
%Turnshek, D.~A., Rao, S.~M., Ptak, A.~F., Griffiths, R.~E., \& Monier,
%E.~M.\ 2003, ApJ, 590, 730

%\bibitem[van de Voort \& Schaye(2011)]{freeke11b} 
%van de Voort, F., \& Schaye, J.\ 2011, arXiv:1111.5039

%\bibitem[van de Voort et al.(2011)]{freeke11a}
% van de Voort, F., Schaye, J., Booth, C.~M., Haas, M.~R., \& Dalla
%Vecchia, C.\ 2011, \mnras, 414, 2458

%\bibitem[Vogt {\etal}(1994)]{vogt94}
%Vogt, S. S., {\etal} 1994, SPIE, 2198, 362

%\bibitem[Weiner {\etal}(2009)]{weiner09} 
%Weiner, B.~J., et al.\ 2009, ApJ, 692, 187

%\bibitem[Wiersma et al.(2009)]{wiersma09} 
%Wiersma, R.~P.~C., Schaye, J., \& Smith, B.~D.\ 2009, \mnras, 393, 99

%\bibitem[Wolfe et al.(2003)]{wolfe03} 
%Wolfe, A.~M., Gawiser, E., \& Prochaska, J.~X.\ 2003, ApJ, 593, 235

%\bibitem[Wolfe et al.(1986)]{wolfe86} 
%Wolfe, A.~M., Turnshek, D.~A., Smith, H.~E., \& Cohen, R.~D.\ 1986, ApJS, 61, 249

%\bibitem[Wolfire et al.(1995)]{wolfire95} 
%Wolfire, M.~G., Hollenbach, D., McKee, C.~F., Tielens, A.~G.~G.~M., \&
%Bakes, E.~L.~O.\ 1995, ApJ, 443, 152

%\bibitem[Whitmore(1995)]{whitmore95}
%Whitmore B., 1995, Photometry with the WFPC2. In: Koratkar A., Leitherer C. (eds.) Calibrating Hubble Space Telescope: Post servicing mission, STScI, Baltimore

%\bibitem[Whyte {\etal}(2002)]{whyte02}
%Whyte, L. F., Abraham, R. G., Merrifield, M. R., Eskridge, P. B., Frogel, J. A., \& Pogge, R. W. 2002, MNRAS, 336, 1281

%\bibitem[White \& Rees(1978)]{white78} 
%White, S.~D.~M., \& Rees, M.~J.\ 1978, MNRAS, 183, 341 

%\bibitem[Whiting et al.(2006)]{whiting06} 
%Whiting, M.~T., Webster, R.~L., \& Francis, P.~J.\ 2006, MNRAS, 368,
%341

%\bibitem[Willmer {\etal}(2005)]{willmer05}
%Willmer, C. N. A., {\etal} 2005, ApJ, submitted (astro--ph/0506041)

%\bibitem[Yanny(1990)]{yanny90}
%Yanny, B. 1990, ApJ, 351, 396

%\bibitem[York {\etal}(1986)]{york86} 
%York, D.~G., Dopita, M., Green, R., \& Bechtold, J.\ 1986, ApJ, 311,
%610

%\bibitem[Zabludoff \& Mulchaey(1998)]{zabludoff98} 
%Zabludoff, A.~I., \& Mulchaey, J.~S.\ 1998, ApJ, 496, 39

%\bibitem[Zonak {\etal}(2004)]{zonak04} 
%Zonak, S. G., Charlton, J. C., Ding, J., \& Churchill, C. W. 2004,
%ApJ, 606, 196

%\bibitem[Zheng {\etal}(2005)]{zheng05} 
%Zheng, X.Z., Hammer, F., Flores, H., Ass\'{e}mat, F., \& Rawat, A. 2005, A\&A, 435, 507

%\bibitem[Zibetti {\etal}(2007)]{zibetti07}
%Zibetti, S., M{\'e}nard, B., Nestor, D.~B., Quider, A.~M., Rao, S.~M., \& Turnshek, 
%D.~A.\ 2007, ApJ, 658, 161 

%\bibitem[Zibetti {\etal}(2005)]{zibetti05} 
%Zibetti, S., M{\'e}nard, B., Nestor, D., \& Turnshek, D.\ 2005, ApJ,
%631, L105

%\bibitem[Zych et al.(2009)]{zych09} 
%Zych, B.~J., Murphy, M.~T., Hewett, P.~C., \& Prochaska, J.~X.\ 2009,
%MNRAS, 392, 1429

%\bibitem[Zwaan et al.(2008)]{zwaan08} 
%Zwaan, M., Walter, F., Ryan-Weber, E., Brinks, E., de Blok, W.~J.~G.,
%\& Kennicutt, R.~C.\ 2008, AJ, 136, 2886


\end{thebibliography}


\end{document}


